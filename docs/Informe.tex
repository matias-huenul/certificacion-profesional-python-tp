\documentclass{article}
\usepackage[spanish]{babel}
\usepackage[a4paper,top=2cm,bottom=2cm,left=3cm,right=3cm,marginparwidth=1.75cm]{geometry}
\usepackage{amsmath}
\usepackage{graphicx}
\usepackage{float}
\usepackage[colorlinks=true, allcolors=blue]{hyperref}
\hypersetup{linkcolor=black}

\title{Certificación Profesional en Python (ITBA) \\
Trabajo Práctico Final}
\author{Matías Huenul}
\date{}
\begin{document}
\maketitle
\tableofcontents
\newpage

\section{Introducción}
En este trabajo práctico se desarrolló un programa de línea de comandos en Python para
obtener información de una API de finanzas, almacenarlas en una base de datos local SQLite3
y graficarlos usando Matplotlib.

\section{Diseño de la aplicación}
La aplicación se estructuró en módulos reutilizables, y un script principal que se encarga
del flujo del programa y la interacción con el usuario.

\subsection{Base de datos}
El módulo \textbf{database} contiene las funciones necesarias para realizar consultas
sobre la base de datos. En este caso, se utilizó el motor \textbf{SQLite3},
y se definió una única base de datos almacenada en el archivo \textbf{tickers.db}.
La misma posee una tabla, \textbf{tickers}, la cual almacena los valores por ticker y fecha, y
tiene la siguiente estructura.

\begin{itemize}
    \item \textbf{symbol (string):} Símbolo que identifica a un ticker.
    \item \textbf{name (string):} Nombre del ticker.
    \item \textbf{value (real):} Valor del ticker.
    \item \textbf{date (string):} Fecha correspondiente al dato.
\end{itemize}

\subsection{Visualizaciones}
El módulo \textbf{plot} contiene una función para realizar gráficos de tipo \textbf{line plot}
utilizando la biblioteca \textbf{Matplotlib}.

\subsection{Interacción con la API de Polygon}
El módulo \textbf{polygon} contiene las funciones para comunicarse con la Rest API de Polygon.
Se define la función \textbf{get\_tickers}, la cual realiza un request
de tipo \textbf{GET} al endpoint \textbf{/v3/reference/tickers/\{ticker\}} y devuelve un diccionario con
los datos de un ticker para una fecha dada. En particular, se devuelven los campos que serán
almacenados en la base de datos, siendo calculado el valor del ticker
como $value = \frac{market\_cap}{weighted\_shares\_outstanding}$, de acuerdo a la documentación oficial
de la API.

\subsection{Utilidades}
El módulo \textbf{utils} posee funciones útiles para el programa, como funciones de cálculo de
fechas, escritura a archivos e interacción con el usuario.

\subsection{Flujo del programa}
La lógica principal del programa se encuentra en el script \textbf{main.py}. El mismo ejecuta
un \textbf{wizard} que interactua con el usuario y lo guía a través de las distintas operaciones
disponibles, las cuales se pueden categorizar en tres grupos: \textbf{actualización},
\textbf{visualización} y \textbf{exportación} de datos.
Si el usuario elige la primera, se debe ingresar el nombre del ticker a consultar, y las fechas
de inicio y fin de la consulta.

\section{Conclusiones}

\bibliographystyle{alpha}
\bibliography{sample}
\end{document}
